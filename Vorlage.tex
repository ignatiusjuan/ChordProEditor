\documentclass{scrbook}
\usepackage[ngerman]{babel}		% deutsche Trennregeln
\usepackage{microtype}				% verbesserter Randausgleich
\usepackage[colorlinks=true, urlcolor=blue]{hyperref}
\usepackage[a4paper,inner=2.5cm,lmargin=2.5cm,outer=2.5cm,tmargin=4cm,bmargin=2.5cm]{geometry}
\usepackage{graphicx}
\usepackage{wrapfig}
\usepackage[]{algorithm2e}

\newcommand*\oldurlbreaks{}		%URLs richtig trennen
\let\oldurlbreaks=\UrlBreaks
\renewcommand{\UrlBreaks}{\oldurlbreaks\do\a\do\b\do\c\do\d\do\e\do\f\do\g\do\h\do\i\do\j\do\k\do\l\do\m\do\n\do\o\do\p\do\q\do\r\do\s\do\t\do\u\do\v\do\w\do\x\do\y\do\z\do\?\do\&}

\author{wir}
\title{Der ChordPro-Editor}
\date{\today}

\begin{document}

\maketitle

\tableofcontents

\chapter{Einleitung}
	\section{Gr\"unde f\"ur den Editor}

\chapter{Grundlagen}
	\section{ChordPro-Dateien}
		\subsection{Syntax}
	
	\section{LaTeX}
		\subsection{Benutzte Syntax}
		
	\section{Implementierung}
	

\chapter{Fazit}
	\section{Vorteile}
	\section{Nachteile}

\chapter{Benutzung}
	\section{Installation}
	
	\section{Anwendung}
		\subsection{Laden}
		
		\subsection{Bearbeiten}
		
		\subsection{Speichern}

\end{document}